\chapter{The Booting Process}
\section{Base System Components}

\subsection{init.elf}
The init process is the first user mode process that is launched and has several duties like launching all
subsequent processes and keeping track of crashed children. The start routine needed looks like in figure
\ref{fig:initprocess}.

%%\begin{enumerate}
%%\item Start vfs.elf
%%\item Wait for the "vfs" service to crop up
%%\item Open config file "init.cfg"
%%\item Read necessary options
%%\item Start all subsequent processes described in the config file
%%\item Wait several seconds
%%\item Check on every child process
%%\item Go two steps back
%%\end{enumerate}

\begin{figure}[h!]
\centering
\scalebox{0.8}{\input{chapters/uml/initprocess.latex}}
\caption{The init process}
\label{fig:initprocess}
\end{figure}

If a child crashes the init process should restart the process in a timely manner so normal operation can
continue. If the broken process keeps crashing in a boot loop the init process needs to decide to
stop relaunching the program.

\subsection{vfs.elf}
The Virtual File System has several responsibilities regarding files, directories and mount points.
It has to implement the interface described in the "Operating System Interfaces" chapter and respond to the
specified messages.

\subsection{*.drv}
All drivers should be launched directly using init.elf so they have all necessary rights to access
advanced kernel functionality. It is possible to start them from any user process as long as the
parent has the User Identification (UID) 0 and thus enough rights to request memory regions, call BIOS interrupts
and do general driver operations. Drivers and driver launch order is specified in the init.cfg file.

\section{The Booting Sequence}
\begin{figure}[h!]
%% \centering
\scalebox{0.70}{\input{chapters/uml/bootsequence.latex}}
\caption{The boot sequence}
\label{fig:bootsequence}
\end{figure}